\chapter{Estándares}
{\justifying
		\section{Estandarizaciones}
		Las estandarizaciones van desde el nombrado de archivos hasta el nombre de las variables y comandos que podemos llegar a usar, así que presentamos algunas recomendaciones en este apartado para mejorar el costo de producción de nuestro código y contenido en el documento \LaTeX.
		\subsection{Estadarizaciones comunes}
			Las estadarizaciones que a continuación presentaremos son tipicas en algunos lenguajes de programación, la idea es adoptarlas en nuestros documentos \LaTeX.
			\subsubsection{UpperCamelCase}
				La escritura utilizando UpperCamelCase es iniciar en mayúscula cada palabra que compone el nombre de la variable, sin olvidar que esta no puede tener espacios ni caracteres especiales. 
			\subsubsection{lowerCamelCase}
				La escritura utilizando lowerCamelCase es iniciar en minúscula cada palabra que compone el nombre de la variable, sin olvidar que esta no puede tener espacios ni caracteres especiales.
			\subsubsection{snake\_case}
				La escritura utilizando snake\_case es usar todo en minúscula y los espacios son reemplazados por guiones bajos. 
			\section{Comandos}
			Recuerde que la estructura de un comando es:
			\begin{center}
				\begin{tcblisting}{boxlatex}
					\newcommand{\<comando>}[<argumentos>]{<función>}
				\end{tcblisting}
			\end{center}
			Para nombrar correctamente los comandos se usará la notación lowerCamelCase.
			\section{Referencias}
			Todas las referencias de la mayor parte de los entornos deben tener un prefijo, esto conlleva a una buena organización y previene el error de referencias iguales para entornos diferentes.
			\begin{boxbasic}[Estándar]
				Algunas referencias no dependen del template cargado algunas de ellas podrían llegar a ser capítulos, ecuaciones y items, por ello se recomienda seguir el siguiente estándar
				\begin{itemize}
					\item Capítulo. \verb|\label{cap:<nombre>}|.
					\item Ecuación. \verb|\label{eqn:<nombre>}|.
					\item Item. \verb|\label{itm:<nombre>}|.
					\item Imagen. \verb|\label{fig:<nombre>}|.
					\item Bibliografía. \verb|\label{bib:<nombre>}| o \verb|\label{book:<nombre>}|.
					\item Tabla. \verb|\label{tab:<nombre>}|.
				\end{itemize}
				\begin{boxbasic}[Nota]
					Para el \verb|<nombre>| utilizaremos la estandarización lowerCamelCase. 
				\end{boxbasic}
			\end{boxbasic}
		\section{Convención de nombres}
		Todos los archivos y carpetas deben ser nombradas con  lowerCamelCase y además seguir las siguientes recomendaciones
		\begin{enumerate}
			\item Los nombres de los archivos .tex utiliza el idioma en el que esta redactado las notas de clase.
			\item Los nombres de los archivos .sage deben ser escritos en ingles.
		\end{enumerate}
	    \section{Sangrado (Indentación)}
	    El contenido de los entornos debe iniciar con una tabulación, así como se muestra en el siguiente código
	    \begin{tcblisting}{boxlatex}
	    	\begin{<entorno>}
	    		Contenido del entorno.
	    	\end{<entorno>}
	    \end{tcblisting}
	    De forma similar en las ecuaciones se realizan las respectivas tabulaciones en el símbolo \& tal como se observa en el siguiente código
	    \begin{tcblisting}{boxlatex}	
	    	\begin{align*}
	    		(a+b)c&=(a+b)\cdot c\\
	    	        &=ac+bc.
	    	\end{align*}
	    \end{tcblisting}
	    \section{SAGE}
	    Gracias a la librería \textbf{sagetex} de \LaTeX\space podemos programar cálculos internos, claro, hay limitaciones a la hora de mezclar SAGE con \LaTeX\space, como por ejemplo una mejor edición de graficas, notación, recorrer arreglos y por supuesto los pasos intermedios en un cálculo determinado, sin embargo, estas limitaciones se pueden mitigar por la creatividad del autor y su conocimiento en SAGE y \LaTeX\space, por tanto debemos tener un código ordenado. 
	    \subsection{Carpetas}
	    Todos los archivos con codigos SAGE puros (archivos con solo código SAGE), deben estar en la carpeta \textbf{programming/sage}, por esta razón en esta carpeta encontramos el archivo \textbf{initial\_variables.sage}.
	    \begin{boxbasic}[Nota]
	    	Los nombres de los archivos dependen del tipo de contenido que poseen, por ejemplo, si su contenido es una clase el nombre del archivo será el mismo que la clase.
	    \end{boxbasic}
	    \subsection{Archivo head.sage}
	    En este archivo estarán las lineas necesarias para cargar las funciones, clases y el archivo \textbf{main.sagetex.tex}, por ejemplo el archivo tendría las siguientes lineas.
	    \begin{tcblisting}{boxlatex}
	    	load("programming/sage/initial_variables.sage")
	    	load("main.sagetex.sage")
	    \end{tcblisting}
%	    \subsection{Convención}
%	    Recordemos que cuando escribimos en el contenido de nuestro documento \LaTeX\space utilizaremos la convención Camel Case por lo que para programar en SAGE utilizaremos la convención del lenguaje \textbf{Python}, es de esperar que fuera así, ya que SAGE esta programado en Python. Para continuar veamos la convención \textbf{snake\_case}, esta convención es fácil de entender, los nombres no tienen caracteres especiales y todo esta en minúscula por ejemplo \textbf{relativity\_theory}. Ahora veamos las convenciones básicas.
%	    \subsubsection{Variables}
%	    Para los nombres de las variables se utilizará la convención snake\_case.
%	    \subsubsection{Funciones}
%	    Para los nombres de las funciones se utilizará la convención snake\_case.
%	    \subsubsection{Clases}
%	    Para los nombres de las clases se utilizará la convención UpperCamelCase.
	    \begin{boxbasic}[Nota]
	    	Si se usa pythontex debe seguirse las mismas pautas dadas para SAGE.
	    \end{boxbasic}
}